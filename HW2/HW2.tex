 \documentclass[11pt]{article}

\usepackage{latexsym}
\usepackage{amssymb}
\usepackage{amsthm}
\usepackage{amscd}
\usepackage{amsmath}
\usepackage{tikz}
\usepackage{graphicx}
\usepackage{enumerate}

\newcommand{\ZZ}{\mathbb{Z}}

\setlength{\evensidemargin}{1in}
\addtolength{\evensidemargin}{-1in}
\setlength{\oddsidemargin}{1.5in}
\addtolength{\oddsidemargin}{-1.5in}
\setlength{\topmargin}{1in}
\addtolength{\topmargin}{-1.5in}

\setlength{\textwidth}{16cm}
\setlength{\textheight}{23cm}

\newcommand{\rook}{\hspace{-.1cm}\amalg\hspace{-.15cm}\bar{}}
\newcommand{\Stab}{\mathrm{Stab}}
\newcommand{\FF}{\mathbb{F}}


\begin{document}
\begin{center}
\section*{William Daniels}
\section*{CSCI 4202}
\subsection*{Artificial Intelligence}
\subsection*{Homework \#2 02/15/15}
\end{center}

\vspace{.25cm}

\textbf{Problem 1:}\\\\
\begin{enumerate}
\item Yes, I do believe it's decomposable, as you can break it down into the distinct problem of simply moving discs from one tower to another, and ignore the fact that you need to get all of them over to a certain tower. 
\item yes, you can always undo the ove by simply reversing yrur steps so far. 
\item Yes, it is predictable, because we can see every possible state. 
\item There is an absolute solution, although there may be many 'relatively good' solutions. Absolute would be definedas the shortest number of derivations until you reach the solution. 
\item I would consider it to be a path to a state since we are concerned with the number of steps, as well as exactly which steps to take, to reach the solution state. 
\item The role of knowledge here tells us about the properties of the curernt state of the problem. 
\item This is most certainly solitary, since there is only one player, and you can base your next move based entirely upon the current state of the problem. \\\\
\end{enumerate}
\hspace{.5cm} \textbf{Problem 2:} \\



\end{document}

